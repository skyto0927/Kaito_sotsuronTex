\documentclass[10ptj,a4j,dvipdfmx,uplatex]{jsbook}

% レイアウト
\usepackage[driver=dvipdfm,hmargin=19.05truemm,vmargin=25.40truemm]{geometry}

% エンコーディング
\usepackage[T1]{fontenc}

% 日本語フォント
\usepackage[expert,deluxe,uplatex]{otf}

% 数式・科学表記
\usepackage{amsmath,amssymb,bm,mleftright,diffcoeff}
\usepackage{siunitx}

% 画像処理
\usepackage[dvipdfmx]{graphicx}

% 参考文献スタイル指定
\usepackage[nobreak]{cite}
\usepackage[resetlabels]{multibib}
\newcites{pubs}{発表文献}
\bibliographystyle{junsrt} %jplain
\bibliographystylepubs{junsrt}
\nocitepubs{*}

% 相互参照
\usepackage[hidelinks]{hyperref}

% URL
\usepackage{url}

% 画像用ディレクトリ
\graphicspath{{images/}}

%%
%% タイトル・著者・日付
%%
\usepackage{dendentitle}
\title{歌唱発声における発声難度の音高・音色依存性に関する分析的検討}
\author{B4 小林海斗}
\date{2020/2/3}

%%
%% main
%%
\begin{document}
\makedendentitle{卒論}{峯松・齋藤研究室}

\chapter{序論}

\begin{table}[h]
\caption{おすすめの執筆順序}% {}内に表題を書く
\begin{center}
\begin{tabular}{|c|c|l|} %セル内の位置{c:センタリング,l:左寄せ},パイプ「|」縦罫線
\hline
1 & 4章 & 開発の中身,実験方法となんとなく見えている(or 予想される)結果 \\
\hline
2 & 3章 & 理論と仮説,この段階ではだいたいでいい,上記実験のベースになっている素案ぐらいでも. \\
\hline
3 & 5章 & 結果のグラフ,理解できるグラフを描きなおすために実験をやり直してもいい.チート厳禁. \\
\hline
4 & 2章+3章 & 関連研究,課題設定→理論までの展開を整理しながら \\
\hline
5 & 1章+6章 & できた結果について素直に受け止められるよう,風呂敷を広げすぎずに. \\
\hline
6 & 論文概要 & 章構成を再度見直し,推敲時にブレないように,ここで固める. \\
\hline
7 & 全章推敲 & このあたりでやっと先輩や先生に見せられるレベル,ただし卒業は見通しが出る. \\
\hline
\end{tabular}
\end{center}
\end{table}

\chapter{関連研究}
過去の先輩の論文を参考にします

%---------------------------%

\chapter{理論的背景}
\section{発声について}
生理的な説明もしてよさそう→開口面積や喉頭を録画する根拠

\section{ピッチ取得について}
ピッチの取得方法には複数の手法があるが、ここでは

%---------------------------%

\chapter{音高・音色における発声難度の調査}
本章では、まず初めに予備実験として音声を収録し、発声難度を主観評価する。
次に、そこで得られた結果からどの要素が発声しにくさとして現れているかを分析する。
\section{予備実験}
\subsection{実験条件}
\begin{description}
    \item[被験者]\mbox{}\\
        実験は20代男性4人を対象として行った。
    \item[収録音声]\mbox{}\\
        発声は/a/、/i/、/u/、/e/、/o/を用いた。
        音高は平均律におけるF3(174.6Hz)、A3(220.0Hz)、C4(261.6Hz)、A4(440.0Hz)の4音を使用した。
    \item[実験内容]\mbox{}\\
        実験内容は以下に示す通りである。
        \begin{description}
            \item[共通項目]\mbox{}\\
                ・全てテンポ120で行う。
                
                ・各実験はそれぞれ2回ずつ行う。
                
                ・発声後に、発声難度を主観で評価する。この評価は「発声しにくい」「どちらとも言えない」「発声しやすい」の3段階で行う。
                
                ・収録と同時に、正面から発声時の口部の様子を録画する。
            \item[実験1]\mbox{}\\
                8拍のカウントの後、/a/の発声でF3の音高を16拍伸ばし8拍休憩する。
                これを音高をA3、C4、F4と変えて行う。
                発声を/i/、/u/、/e/、/o/に変えた場合においても収録する。
                発声が5通り、音高が4通りで合計20回収録する。
            \item[実験2]\mbox{}\\
                4拍のカウントの後、/a/の発声でF3の音高を2拍伸ばし、その後/a/の発声でA3の音高を2拍伸ばす。
                これを、後半の音高A3をC4、F4と変えて行う。
                前後の発声を/i/、/u/、/e/、/o/に変えた場合においても収録する。
                発声が25通り、音高変化が3通りで合計75回収録する。
            \item[実験3]\mbox{}\\
                4拍のカウントの後、/a/の発声でA3の音高を2拍伸ばし、その後/a/の発声でF3の音高を2拍伸ばす。
                これを、前半の音高A3をC4、F4と変えて行う。
                前後の発声を/i/、/u/、/e/、/o/に変えた場合においても収録する。
                発声が25通り、音高変化が3通りで合計75回収録する。
        \end{description}

    \item[使用機材]\mbox{}\\
        ・マイク

        ・カメラ

        ・オーディオIF

        などなど
\end{description}

\subsection{実験結果}
ここに実験結果のグラフ。(時間-F0 or centグラフ)

単音…
立ち上がりで音程が安定しない

上昇下降…
音の切り替え時に安定しない

被験者ごとの比較

\section{得られた結果と発声難度の関連性をがんばって見つけたいパートです}
\subsection{概要}
\subsection{実験手法}
\subsection{実験結果}

目的と実験設計


%---------------------------%

\chapter{自動歌詞生成}
上記を踏まえ、プログラムをまわした結果をかく

%---------------------------%

\chapter{結論}

%---------------------------%

\chapter{謝辞}
本研究を進めるにあたって、齋藤大輔講師には指導教員としてテーマをいただいたほか、研究を通して指導していただきました。深く感謝申し上げます。
峯松信明教授には、もう一人の指導教員として研究の方針について指導いただきました。深く感謝申し上げます。
峯松・齋藤研究室の先輩方にも様々な助言をいただきました。深く感謝申し上げます。
また、忙しい時期にもかかわらず実験に参加してくださった皆様のおかげで研究を進めることができました。深く感謝申し上げます。
最後に、私を支えてくれた家族と友人に深く感謝申し上げます。

\begin{flushright}
2020年2月7日

小林 海斗
\end{flushright}

\bibliography{references}

\end{document}