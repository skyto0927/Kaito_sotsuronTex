\documentclass[10ptj,a4j,dvipdfmx,uplatex, draft]{jsbook}%ここのdraft消すと画像が出る

% レイアウト
\usepackage[driver=dvipdfm,hmargin=19.05truemm,vmargin=25.40truemm]{geometry}

% エンコーディング
\usepackage[T1]{fontenc}

% 日本語フォント
\usepackage[expert,deluxe,uplatex]{otf}

% 数式・科学表記
\usepackage{amsmath,amssymb,bm,mleftright,diffcoeff}
\usepackage{siunitx}


\usepackage{comment}

% 画像処理
\usepackage[dvipdfmx]{graphicx}

% 参考文献スタイル指定
\usepackage[nobreak]{cite}
\usepackage[resetlabels]{multibib}
\newcites{pubs}{発表文献}
\bibliographystyle{junsrt} %jplain
\bibliographystylepubs{junsrt}
\nocitepubs{*}

% 相互参照
\usepackage[hidelinks]{hyperref}

% URL
\usepackage{url}

% 画像用ディレクトリ
\graphicspath{{images/}}

%%
%% タイトル・著者・日付
%%
\usepackage{dendentitle}
\title{歌唱発声における発声難度の音高・音色依存性に関する分析的検討}
\author{B4 小林海斗}
\date{2020/2/3}

%%
%% main
%%
\begin{document}
\makedendentitle{卒論}{峯松・齋藤研究室}

\chapter{序論}

\begin{table}[h]
\caption{おすすめの執筆順序}% {}内に表題を書く
\begin{center}
\begin{tabular}{|c|c|l|} %セル内の位置{c:センタリング,l:左寄せ},パイプ「|」縦罫線
\hline
1 & 4章 & 開発の中身,実験方法となんとなく見えている(or 予想される)結果 \\
\hline
2 & 3章 & 理論と仮説,この段階ではだいたいでいい,上記実験のベースになっている素案ぐらいでも. \\
\hline
3 & 5章 & 結果のグラフ,理解できるグラフを描きなおすために実験をやり直してもいい.チート厳禁. \\
\hline
4 & 2章+3章 & 関連研究,課題設定→理論までの展開を整理しながら \\
\hline
5 & 1章+6章 & できた結果について素直に受け止められるよう,風呂敷を広げすぎずに. \\
\hline
6 & 論文概要 & 章構成を再度見直し,推敲時にブレないように,ここで固める. \\
\hline
7 & 全章推敲 & このあたりでやっと先輩や先生に見せられるレベル,ただし卒業は見通しが出る. \\
\hline
\end{tabular}
\end{center}
\end{table}


\begin{comment}
\chapter{中間}
\section{はじめに}
近年、カラオケが一般的な娯楽となっており、歌を歌うことも身近なものとなっている。
人間は歌う際に、声の高さを適切に制御することで特定の旋律を表現する。しかし、
出せる声の高さには限界がある。また、
高音域を歌う際に母音が「i」だと途端に歌いにくくなるという経験は想像に難くない。
歌詞の点からみてもサビにおける高音の「i」の頻度は「a」に比べて低く\cite{popular_highest}、作詞段階で意図的に
避けられていると考えられる。
この発声難度の違いの定量的な知見が得られれば、歌詞作成など様々な応用が考えられる。
しかし、この違いはこれまで定量的に分析されていなかった。そこで、ここでは
この違いを定量的に分析する様々な手法について検討する。
分析にはいくつかの手法があるが、ここでは母音と関連の深いフォルマント周波数や、音色を
表す声道特徴量に着目する。
また、人間の声の高さの物理的制約にも分析するため、声帯の機構についても検討する。

第2章では母音とフォルマント周波数がどのように関連しているか説明し、
第3章では音高と声帯周辺の筋肉の関係を説明する。
そして第4章でいくつかの分析方法を示し、第5章で研究の方針を示す。

\section{母音とフォルマント周波数}
母音の音声は、肺からの空気が声帯を振動させ、声道を通過して音響的に整形されることで生じる。
ここで声帯振動による音を喉頭音源とよぶ。
声道は共鳴器であり、フォルマント周波数周辺の音は他よりも強い音となって口唇から伝播する。

声道の形状を、声門への距離に応じて変化する断面積の関数として考えると、声道の形状は曲線として記述できる。
これを声道断面積関数とよぶ\cite{science}。
\begin{figure}[htbp]
    \begin{center}
      \includegraphics[clip,width=12.0cm]{声道断面積関数.png}
      \caption{断面積関数\cite{science}}
      \label{fig:danmen}
    \end{center}
  \end{figure}

フォルマント周波数は母音や声質の決定において非常に重要である。
フォルマント周波数は声道スペクトルのピークとなる共鳴周波数のことであり、
低い方から順に第1フォルマント(F1)、第2フォルマント(F2)、と名付けられる。

調音器官の運動は、フォルマント周波数全てに影響を与える。
第1フォルマントは顎の開きに敏感で、開き具合が大きくなるほど第1フォルマント周波数は上昇する。
第2フォルマントは特に舌の形状に大きく影響を受ける。
声道前方を狭めるときに第2フォルマント周波数はもっとも高くなる。
逆に、軟口蓋や咽頭部を狭める時には低くなる。
第3フォルマントは舌尖の位置、正確には前歯のすぐ後ろの空間に影響を受ける。
空間が大きいほど第3フォルマント周波数は低くなる。

フォルマント周波数のうち特にF1、F2は母音を特徴付ける重要なパラメータであり、これらの分布によって母音を分類することができる\cite{japanese_vowels}。
図\ref{fig:jp_formant}は日本語5母音の第1、第2フォルマント周波数を示したものである。

\begin{figure}[htbp]
    \begin{center}
      \includegraphics[clip,width=7.0cm]{5母音.png}
      \caption{日本語5母音の第1、第2フォルマント周波数\cite{japanese_vowels}}
      \label{fig:jp_formant}
    \end{center}
  \end{figure}
  ここでAMは成人男性、AFは成人女性、ADは青年、Cは子供である。一般に、平均的な声が高いほど
  フォルマント周波数も高い傾向にある。
  成人男性を例に挙げると、第1フォルマントは250〜1000 \si{Hz}、第2フォルマントは600〜2500 \si{Hz}、
  第3フォルマントは1700〜3500 \si{Hz}である\cite{science}。





母音「a」では口唇側が大きく、声門側が小さい一方、母音「i」では口唇側が小さく、声門側が大きい。
また、母音「a」のF1は高く、F2は低い一方、母音「i」のF1は低く、F2は高い。
このように、声道形状とF1、F2の関係はよく知られており、舌の高さとF1、舌の前後方向の位置とF2にはそれぞれ相関がある\cite{katei}。

調音器官には口唇、舌の他に顎、喉頭なども含まれる。これらは連動して動くため、1つの調音器官のみを変化させて考えるのは適切でない。
\begin{figure}[htbp]
    \begin{center}
      \includegraphics[clip,width=7.0cm]{声道形状.png}
      \caption{声道の輪郭\cite{science}}
      \label{fig:seido}
    \end{center}
\end{figure}

\section{喉頭音源の制御と音高}
発声周波数は、声帯の張力と厚みによって決まる。ここで張力は声帯長を操作して変化する。そのため、
ピッチが低い時声帯は厚く短くなっており、逆にピッチが高い時声帯は薄く長くなっている。
この声帯長の変化は、主に輪状甲状筋の収縮によって引き起こされる(図\ref{fig:rinjo})。

\begin{figure}[htbp]
    \begin{center}
      \includegraphics[clip,width=10.0cm]{輪状甲状筋.png}
      \caption{輪状甲状筋の機能\cite{science}}
      \label{fig:rinjo}
    \end{center}
\end{figure}

輪状甲状筋は直部と斜部に分かれている。
直部の収縮は輪状軟骨を傾け、それにより輪状軟骨と甲状軟骨の距離を短くする。
その結果甲状軟骨と披裂軟骨との距離が長くなり、声帯を伸展させる。
斜部の収縮は、輪状軟骨を前方に動かし、甲状軟骨と輪状軟骨の距離を大きくすることで声帯を伸展させる。

また、喉頭は「i」のように口唇を横に引っ張って発声する時に上昇し、
「u」のように円唇母音の時に下降する。
喉頭の位置は発声周波数によっても変化し、一般に周波数が高いほど喉頭の位置も高くなる(図\ref{fig:koto})。

\begin{figure}[htbp]
    \begin{center}
      \includegraphics[clip,width=12.0cm]{喉頭位置.png}
      \caption{男性歌手(a)と歌手でない男性(b)における、発声周波数ごとの垂直方向の喉頭の位置\cite{science}}
      \label{fig:koto}
    \end{center}
\end{figure}

歌手でない男性の喉頭位置の変化(b)を見ると、位置は人により様々であるが、200 \si{Hz}までは喉頭位置が発声周波数の上昇に
したがって上昇している。
(a)はプロの歌手の結果であり、(b)と比べると大きく異なっている。一般に喉頭位置が発声周波数の上昇に伴って
下降しているといえる。
このように喉頭の位置は発声周波数にある程度関係しているが、意図的にコントロールすることも可能である。



\section{声道特徴量の分析手法}

音声は、喉頭音源と声道によって生成される。音色の特性に着目するには声道特徴量も重要である。
母音ごとの特徴を見るのであればこれらの特性を分離すると都合が良い。
声道特徴量の分析にはケプストラム分析やLPC分析がある。これらについて説明していく。

\subsection{ケプストラム分析}
ケプストラム分析とは、喉頭音源特性と声道特性を分離して、スペクトル包絡を求める手法である。
まず音声波形データに下処理(後述)を施したのちフーリエ変換する。
次に、各周波数量の対数をとってデシベル表記に変換する。
これを逆フーリエ変換しケプストラム領域に移すことで、喉頭音源特性と声道特性を分離できる。
ここで低次のケプストラムは声道特性(スペクトル包絡)に、
高次のケプストラムは喉頭音源特性(スペクトル微細構造)に対応しているため、
低次のみを取り出すことで包絡を抽出することができる。

これは以下のように説明できる。
喉頭音源を$g(t)$、声道断面積関数を$h(t)$とすると、出力信号は$g(t)$、$h(t)$の畳み込み積分
\begin{equation}
    x(t) = g(t) * h(t)
\end{equation}
とかける。
これは周波数領域では$G(\Omega)$、$H(\Omega)$を用いて、
\begin{equation}
    X(\Omega) = G(\Omega) \times H(\Omega)
\end{equation}
とかける。対数をとることで
\begin{equation}
    \log |X(\Omega)| = \log |G(\Omega)| + \log |H(\Omega)|
\end{equation}
となり、確かに分離できることがわかる。


なお、FFTを行う前に下処理として高域強調処理、窓関数適用を行う。

\subsubsection{高域強調処理}
音声パワーは広域になるほど減衰するため、高域強調処理を行う。一般には6 \si{dB/oct}の強調となるように行う。
高域通過フィルタとしては以下のような1次有限インパルス応答フィルタ$H(z)$が用いられる。
\begin{equation}
    H(z) = 1 - \alpha z^{-1}
\end{equation}

ここで$z= \exp(j\omega)$、$\omega = 2\pi f/f_s$で、$f$は周波数、$f_s$はサンプリング周波数である。

量子化された離散信号においては
\begin{equation}
    y(n) = x(n) - \alpha x(n-1)
\end{equation}
と表す。$\alpha$ の値としては0.97がよく用いられる。

\subsubsection{窓関数}
FFTは周期関数に対して適用するが、一般の音声データは周期的になっておらず両端が不連続であり、
このまま分析を行うと高周波に雑音が乗ってしまう。
そこで、波形に対して両端が減衰するように分析窓をかける。
分析窓としてはハミング窓とハニング窓が知られており、それぞれ以下の式で表せる。
\begin{eqnarray}
    W_H (n) &=& 0.54 - 0.46 \cos \frac{2n\pi}{N-1} \\
    W_N (n) &=& 0.5 - 0.5 \cos \frac{2n\pi}{N-1}
\end{eqnarray}

ただし$n$が0未満あるいは$N$以上の時はゼロの値をとる。
多くの場合は前者のハミング窓が用いられる。

\subsubsection{MFCC}
MFCC(メル周波数ケプストラム係数)は人間の聴覚の特性に基づく分析手法であり、主に音声認識分野で使われる\cite{onsei}。

人間は音が高くなるにつれて音高の変化を小さく感じてしまう。そこで用いられる尺度がメル尺度であり、
1000 \si{Hz}を基準としてその$n$倍に知覚する周波数を$n\times 1000$ \si{mel}と表す。
いくつかの種類のうち、よく使われるのが以下のファントの式である。
\begin{equation}
    m = \frac{1000}{\log_{10} 2} \log_{10} \left( \frac{f}{1000}+1 \right)
\end{equation}

これを適用するために、メルフィルタバンクをかける。
これはファントの式に合わせた尺度で周波数軸をとり、その上で幅が均等になるようにフィルタを割り振っている(図\ref{fig:mfcc})。
\begin{figure}[htbp]
    \begin{center}
      \includegraphics[clip,width=7.0cm]{MFCC.png}
      \caption{メルフィルタバンク\cite{onsei}}
      \label{fig:mfcc}
    \end{center}
\end{figure}

これ以降の手順は上記のケプストラム分析と同様である。

\subsection{LPC分析}
ケプストラム分析とは別の手法として、線形予測分析(LPC)がある\cite{lpc}。
まず声道フィルタを、極のみをもち零をもたないフィルタで近似する。
\begin{equation}
    H(z) = \frac{1}{\prod^p _{i=1} \left( 1-\frac{z}{z_i} \right)} 
    = \frac{1}{\sum_{k=0}^p \alpha_k z^{-k}}
\end{equation}

ただし、$\alpha_0 = 1$である。伝達関数の式$H(z) = G(z)H(z)$に代入すると、
\begin{equation}
    X(z) \left( \sum_{k=0}^p \alpha_k z^{-k} \right) = G(z)
\end{equation}
となり、両辺を逆Z変換すると、
\begin{equation}
    x[n] + \alpha_1 x[n-1] + \cdots + \alpha_p x[n-p] = g[n]
\end{equation}
となる。したがって、音声信号は
\begin{equation}
    x[n] = -\sum_{k=1}^p \alpha_k x[n-k] + g[n]
\end{equation}
と表せる。

右辺第1項を$\hat{x}_p[n]$と置き、第2項を$\epsilon[n]$と置き換えると、
\begin{equation}
    x[n] = \hat{x}_p[n] + \epsilon[n]
\end{equation}
となる。これは、現在の出力が過去の出力の線型結合$\hat{x}_p[n]$と残差$\epsilon[n]$の和で書けることを意味する。
線形予測分析はこの誤差のパワーが小さくなるように$\alpha_k$の値を調整し、スペクトルを近似することだといえる。

\subsection{分析手法の比較}

ケプストラム分析とLPC分析を比較したものが図\ref{fig:hikaku}である。

\begin{figure}[htbp]
    \begin{center}
      \includegraphics[clip,width=7.0cm]{比較.png}
      \caption{分析結果の比較\cite{lpc}}
      \label{fig:hikaku}
    \end{center}
\end{figure}

ケプストラム分析では音声の山も谷も同様に平滑化しているが、LPCでは山をより重視したものとなっていることがわかる。
この違いを把握した上で、適切な分析方法を選択する必要がある。

\end{comment}



\chapter{関連研究}
過去の先輩の論文を参考にします

%---------------------------%

\chapter{理論的背景}
\section{発声について}
生理的な説明もしてよさそう→開口面積や喉頭を録画する根拠

\section{ピッチ取得について}
ピッチの取得方法には複数の手法があるが、ここでは音声分析合成システムであるWORLD\cite{world}を用いて行った。

%---------------------------%

\chapter{音高・音色における発声難度の調査}
本章では、まず初めに予備実験として音声を収録し、発声難度を主観評価する。
次に、そこで得られた結果からどの要素が発声しにくさとして現れているかを分析する。
\section{予備実験}
\subsection{実験条件}
\begin{description}
    \item[被験者]\mbox{}\\
        実験は20代男性4人を対象として行った。
    \item[収録音声]\mbox{}\\
        発声は/a/、/i/、/u/、/e/、/o/を用いた。
        音高は平均律におけるF3(174.6Hz)、A3(220.0Hz)、C4(261.6Hz)、A4(440.0Hz)の4音を使用した。
    \item[実験内容]\mbox{}\\
        実験内容は以下に示す通りである。
        \begin{description}
            \item[共通項目]\mbox{}\\
                ・全てテンポ120で行う。
                
                ・各実験はそれぞれ2回ずつ行う。
                
                ・発声後に、発声難度を主観で評価する。この評価は「発声しにくい」「どちらとも言えない」「発声しやすい」の3段階で行う。
                
                ・収録と同時に、正面から発声時の口部の様子を録画する。
            \item[実験1]\mbox{}\\
                8拍のカウントの後、/a/の発声でF3の音高を16拍伸ばした後8拍休憩する。
                これを音高をA3、C4、F4と変えて行う。
                発声を/i/、/u/、/e/、/o/に変えた場合においても同様に行う。
                発声が5通り、音高が4通りで合計20回収録する。
            \item[実験2]\mbox{}\\
                4拍のカウントの後、/a/の発声でF3の音高を2拍伸ばし、その後/a/の発声でA3の音高を2拍伸ばす。
                これを、後半の音高A3をC4、F4と変えて行う。
                前後の発声を/i/、/u/、/e/、/o/に変えた場合においても同様に行う。
                発声が25通り、音高変化が3通りで合計75回収録する。
            \item[実験3]\mbox{}\\
                4拍のカウントの後、/a/の発声でA3の音高を2拍伸ばし、その後/a/の発声でF3の音高を2拍伸ばす。
                これを、前半の音高A3をC4、F4と変えて行う。
                前後の発声を/i/、/u/、/e/、/o/に変えた場合においても同様に行う。
                発声が25通り、音高変化が3通りで合計75回収録する。
        \end{description}

    \item[使用機材]\mbox{}\\
        ・マイク

        ・カメラ: SONY FDR-AX45

        ・オーディオIF

        などなど
\end{description}

\subsection{実験結果}
ここに実験結果のグラフ。(時間-F0 or centグラフ)


単音…
立ち上がりで音程が安定しない

上昇下降…
音の切り替え時に安定しない

被験者ごとの比較

\section{得られた結果と発声難度の関連性をがんばって見つけたいパートです}
\subsection{概要}
\subsection{実験手法}
\subsection{実験結果}

目的と実験設計


%---------------------------%

\chapter{自動歌詞生成}
上記を踏まえ、プログラムをまわした結果をかく

%---------------------------%

\chapter{結論}

%---------------------------%

\chapter{謝辞}
本研究を進めるにあたって、齋藤大輔講師には指導教員としてテーマをいただいたほか、研究を通して指導していただきました。深く感謝申し上げます。
峯松信明教授には、もう一人の指導教員として研究の方針について指導いただきました。深く感謝申し上げます。
峯松・齋藤研究室の先輩方にも様々な助言をいただきました。深く感謝申し上げます。
また、忙しい時期にもかかわらず実験に参加してくださった皆様のおかげで研究を進めることができました。深く感謝申し上げます。
最後に、私を支えてくれた家族と友人に深く感謝申し上げます。

\begin{flushright}
2020年2月7日

小林 海斗
\end{flushright}

\bibliography{references}

\end{document}