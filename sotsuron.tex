\documentclass[10ptj,a4j,dvipdfmx,uplatex]{jsbook}

% レイアウト
\usepackage[driver=dvipdfm,hmargin=19.05truemm,vmargin=25.40truemm]{geometry}

% エンコーディング
\usepackage[T1]{fontenc}

% 日本語フォント
\usepackage[expert,deluxe,uplatex]{otf}

% 数式・科学表記
\usepackage{amsmath,amssymb,bm,mleftright,diffcoeff}
\usepackage{siunitx}

% 画像処理
\usepackage[dvipdfmx]{graphicx}

% 参考文献スタイル指定
\usepackage[nobreak]{cite}
\usepackage[resetlabels]{multibib}
\newcites{pubs}{発表文献}
\bibliographystyle{junsrt} %jplain
\bibliographystylepubs{junsrt}
\nocitepubs{*}

% 相互参照
\usepackage[hidelinks]{hyperref}

% URL
\usepackage{url}

% 画像用ディレクトリ
\graphicspath{{images/}}

%%
%% タイトル・著者・日付
%%
\usepackage{dendentitle}
\title{進捗報告}
\author{B4 小林海斗}
\date{2020/1/1}

%%
%% main
%%
\begin{document}
\makedendentitle{卒論}{峯松・齋藤研究室}

\chapter{序論}

\begin{table}[h]
\caption{おすすめの執筆順序}% {}内に表題を書く
\begin{center}
\begin{tabular}{|c|c|l|} %セル内の位置{c:センタリング,l:左寄せ},パイプ「|」縦罫線
\hline
1 & 4章 & 開発の中身,実験方法となんとなく見えている(or 予想される)結果 \\
\hline
2 & 3章 & 理論と仮説,この段階ではだいたいでいい,上記実験のベースになっている素案ぐらいでも. \\
\hline
3 & 5章 & 結果のグラフ,理解できるグラフを描きなおすために実験をやり直してもいい.チート厳禁. \\
\hline
4 & 2章+3章 & 関連研究,課題設定→理論までの展開を整理しながら \\
\hline
5 & 1章+6章 & できた結果について素直に受け止められるよう,風呂敷を広げすぎずに. \\
\hline
6 & 論文概要 & 章構成を再度見直し,推敲時にブレないように,ここで固める. \\
\hline
7 & 全章推敲 & このあたりでやっと先輩や先生に見せられるレベル,ただし卒業は見通しが出る. \\
\hline
\end{tabular}
\end{center}
\end{table}

\chapter{関連研究}
過去の先輩の論文を参考にします

\chapter{理論的背景}
\section{発声について}
生理的な説明もしてよさそう→開口面積や喉頭を録画する根拠

\section{歌詞生成について}

\chapter{音高や音程、母音における発声難度の調査}
予備実験と本実験のことを書く

\section{予備実験}
\subsection{概要}
\subsection{実験手法}
母音ごとに音高を変えて収録した。
フォルマント周波数の平均値をとり、また発声しやすさを主観評価で測定した。
フリーソフトウェア「Praat」を使用。
音高は平均律におけるF3(174.6Hz)からA4(440.0Hz)の10音を使用。
F1、F2、F3それぞれの平均値を測定。
測定者は20代男性1名であり、ある程度歌唱経験がある。
発声しにくさは「発声しにくい」、「やや発声しにくい」、「やや発声しやすい」、「発声しやすい」の4段階で評価する。
最も出しやすい音高で「あ」を発声した時を「発声しやすい」と定義した。
\subsection{実験結果}

\section{本実験}
\subsection{概要}
\subsection{実験手法}
以下に実験の手法を示す。
実験は歌唱に普段から親しんでいる男性2人、そうでない男性2人の計4人を対象として行った。
収録する音声はすべて日本語の母音「あ」「い」「う」「え」「お」を用いて行う。
音声は単音発声と2音間の上昇、下降について収録する。
音高については男性の地声の範囲内に収まるよう、F3、A3,C4,F4の4音を採用した。
録音の際ガイドとなる音声が必要なため、短いガイド音声を作成して利用した。
できるだけ正しい音程で収録できるよう、使用する音源を録音前に何度か聞いてもらい、音程が取れたところで収録に移った。

同時に、発声の様子を正面から録画し、口の開口面積および顎、喉頭の下降度合いを計測する。

単音は音声1を用いて、F3,A3,C4,F4を続けて発声する。母音を変えて5回収録する。
上昇は音声2を用いて、F3→A3,F3→C4,F3→F4を続けて発声する。上昇前後ともに母音を変えて25回収録する。
下降は音声3を用いて、A3→F3,C4→F3,F4→F3を続けて発声する。下降前後ともに母音を変えて25回収録する。
以上の工程を2回繰り返す。

\subsection{実験結果}


\chapter{自動歌詞生成}
上記を踏まえ、プログラムをまわした結果をかく

\chapter{結論}

\chapter{謝辞}



\bibliography{references}

\end{document}